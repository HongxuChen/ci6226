\section{Discussion}\label{sec:discuss}
This section contains some of our thoughts on our currently applied approach.

The indexing for Project 1 now takes lots of time (around 130s). That probably results from 1) the one-by-one document indexing strategy; 2) the way we tokenize and store adequate information for most fields ; 3) the additional ``ALL'' field. Basically this is a balance between the indexing efficiency and query accuracy.  The one-by-one indexing can be changed to batching indexing on a more powerful machine where JVM is allowed to allocate more memories.

During the implementation of Top-$N$ topic discovery, we did not do a lemmatization pro-processing on the generated tokens. This means conjugates are not merged as expected. In the future we are planning to add such a feature. Additionally, for this particular case, part-of-speech tagger~\cite{pos_tagger} methodologies might be used to refine the topic results due to the fact that topics are typically nouns with possible preceding adjectives or nouns (e.g., ``machine learning'', ``neural networks''). Although did not appear in most of our searches, there exist some cases that unusual combination of N-Grams (e.g., \textsf{wireless sensor}) should be filtered out.  We actually tried POS tagger~\footnote{Stanford CoreNLP, \url{http://nlp.stanford.edu/software/tagger.shtml\#Download}} to refine the generated N-Grams, however it either took too much time (e.g., over one hour) or caused a Memory-Out-Of-Bound exception, therefore we finally discarded this improvement.

Top-$N$ topic discovery and top-$N$ similar venue discovery are tightly related. The way we treat all publication in a special venue and year is enlightening since this actually accumulates several naturally relevant content; as the experiments show, the similarity based \textsf{MoreLikeThis} query is precise enough. In order to achieve more refined top-$N$ topic results, we perhaps should do some classification  according to the venue information.

In our experiment, we used the indexed document in Project 1 for Application 1. But since only the title information is used, we can base the top-N topic discovery on virtual document indexing; and this will potentially have a performance gain.